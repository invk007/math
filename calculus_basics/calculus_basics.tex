\documentclass[12pt, a4paper]{scrartcl}
\usepackage[utf8]{inputenc}
\usepackage[english]{babel}
\usepackage{graphicx}
\usepackage{amsmath}
\usepackage[left=2cm, right=1.5cm, top=2cm, bottom=2cm]{geometry}
\setlength{\parindent}{0pt}
\author{Ilya Novikau}
\title{Calculus Basics}
\begin{document}

\maketitle

\tableofcontents

\newpage

\section{Limits}
\label{sec:limits}

\subsection{Definition}
\label{sec:limits:def}
$\lim_{x \to a}f(x) = L$ \textbf{if and only if} for any $\epsilon > 0$ there is $\delta > 0$ such that $|f(x) - L| < \delta$.

\subsection{Infinite Limit}
\label{sec:limits:inf_limit}
For every $M > 0$ ($M$ is very big number) there exists $\delta > 0$ such that for all $x$ if $0 < |x - a| < \delta$ then $f(x) > M$.

\subsection{Squeeze theorem}
\label{sec:limits:squeeze_theor}
Regardless of what $y = f(x)$ looks like if $$g(x) <= f(x) <= h(x)$$ and $$\lim_{x \to a}g(x)=\lim_{x \to a}h(x) = L$$ then $$\lim_{x \to a}f(x) = L$$

\subsection{Intermediate value theorem}
\label{sec:limits:ivt}
If $f(x)$ is \textbf{continuous} and the graph passes through $y = A$ and $y = C$ where $A < C$, then it also \textbf{has to pass} through $y = B$ for $A < B < C$.

\subsection{Continuity of function}
\label{sec:limits:continuity}
A function $f$ is \textbf{continuous} at a point $a$ if $\lim_{x \to a}f(x)$ exists and equals to $f(a)$.

\section{Derivatives}
\label{sec:derivatives}

\subsection{Definition}
\label{sec:derivatives:def}
The derivative of $f(x)$ at the point $a$ is defined to be the limit: $$\lim_{x \to a}\dfrac{f(x) - f(a)}{x - a}$$

Derivative also expressed as a limit as the change $h$ in $x$ goes to $0$: $$\lim_{h \to 0}\dfrac{f(h+a) - f(a)}{h}$$

Views of derivatives:
\begin{enumerate}
    \item \textbf{Graphical view}: the derivative is the slope of the tangent line to the graph of the function at $a$.
    \item \textbf{Input-nudging view}: the derivative tells us, when $x$ is nudged a little, by what factor of that change the outputs of $f$ will be changed.
    \item \textbf{Symbolic view}: derivative of functions given by formulas: polynomials, trig functions and more.
\end{enumerate}


\subsection{Mean Value Theorem}
\label{sec:derivatives:mvt}

The \textit{instantaneous} rate of change (i.e. derivative) is a limit of the \textit{average} rate of change over smaller and smaller intervals.\\
\textbf{Theorem}: If the average rate of change of $f(x)$ between $x = a$ and $x = b$ is $s$, then there must be \textit{at least one} point $c$ between $a$ and $b$ where $f'(x)=s$

\subsection{Concavity}
\label{sec:derivatives:concavity}

The graph $y = f(x)$ is concave up at $a$ if and only if  $f''(a) > 0$.	It's concave down at $a$ if and only if $f''(a) < 0$.

\subsection{Computing Derivatives}
\label{sec:derivatives:computing}

\subsubsection{Product Rule}
\label{sec:derivatives:computing:product}

$$(fg)'(a)=f'(a)g(a) + f(a)g'(a)$$

\subsubsection{Division Rule}
\label{sec:derivatives:computing:division}

$$(\dfrac{1}{f(a)})' = \dfrac{-f'(a)}{f(a)^2}$$

\subsubsection{Quotient}
\label{sec:derivatives:computing:quotient}

$$(\dfrac{f(x)}{g(x)})'=\dfrac{f'(x)g(x) - f(x)g'(x)}{g(x)^2}$$

\subsection{Exponentials and Logarithms}
\label{sec:derivatives:exp}
$$(e^x)' = e^x$$
$$(\ln x)' = \dfrac{1}{x}$$

\section{Linear Approximation and Applications}
\label{sec:applications}

\subsection{Approximation}
\label{sec:application:approximation}

It is sometimes very useful to replace a complicated function $f$ by its \textit{linear approximation} $L$.\\
Linear approximation of function $f(x)$ in a point $a$.
$$L(x) = f'(a)(x-a) + f(a)$$

\subsubsection{Taylor polynomials}
\label{sec:application:approximation:taylor_polynomials}
The formula for the linear approximation to $f$ at 0 is $$L(x) = f'(0)x + f(0)$$
For $L$ is true that \textit{$L$ and $f$ have the same value at 0 and the same derivative at 0}.\\
Proof: \begin{enumerate}
    \item $L(0) = f'(0) \cdot 0 + f(0) = f(0)$
    \item $L'(x) = (f'(0)x + f(0))' = f'(0)$
\end{enumerate}
\textbf{Taylor polynomials} can be used to approximate a function better and better.\\
The linear approximation to a function at 0 is: $$y=f'(0)x + f(0)$$
The quadratic approximation to a function at 0 is $$y=\dfrac{f''(0)}{2}x^2 + f'(0)x + f(0)$$
The cubic approximation is $$y = \dfrac{f'''(0)}{3!}x^3 + \dfrac{f''(0)}{2}x^2 + f'(0)x + f(0)$$

\subsection{Application: Pendulums}
\label{sec:application:pendulums}

\newpage

\appendix

\section{Algebra}
\label{appendix:algebra}

\subsection{Line equation}
\label{appendix:algebra:line_equation}

By definition \textit{slope} is $\dfrac{\Delta y}{\Delta x}$\\
Point-slope form of line equation: $$y - y_0 = m (x - x_0)$$
Slope-intercept form of line equation: $$y = mx + b$$

\end{document}
